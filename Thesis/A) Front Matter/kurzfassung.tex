\chapter{Kurzfassung}

Die Struktur von Molekülwolken ist entscheidend für das Verständnis der physikalischen Prozesse, die die Sternentstehung steuern. In dieser Arbeit präsentieren wir eine detaillierte morphologische und topologische Analyse der Molekülwolken Orion~A und Orion~B mithilfe von Methoden der Fraktalgeometrie und Minkowski-Funktionale. Auf Basis von Säulendichtekarten, die aus Staubemissionen im fernen Infrarot gewonnen wurden, bestimmen wir sowohl globale als auch lokale fraktale Dimensionen über einen breiten Dynamikbereich hinweg.

Unsere Ergebnisse zeigen in Orion~A einen strukturellen Übergang bei einer Säulendichte von $N \sim 1{,}2 \times 10^{22} \,\mathrm{cm}^{-2}$, der mit dem Auftreten dichter Kerne und erhöhter Sternentstehungsaktivität zusammenfällt. Orion~B hingegen weist eine durchgängig skalenfreie, turbulente Morphologie ohne einen solchen Übergang auf. Die Euler-Charakteristik untermauert diese Interpretation durch ein deutliches topologisches Maximum in Orion~A, das in Orion~B fehlt. 

Zusätzlich führen wir eine lokale Karte der fraktalen Dimension sowie sogenannte Masse–Größe–Dimensions–Diagramme (MSD) ein, um die strukturelle Komplexität ortsabhängig zu analysieren. Ein Vergleich mit der Verteilung junger Sternobjekte (YSOs) zeigt, dass Regionen mit höherer fraktaler Komplexität insbesondere in Orion~A vermehrt frühe Entwicklungsstadien der Sternentstehung aufweisen. Dies deutet auf eine Verbindung zwischen komplexer Morphologie und aktiver Sternbildung hin.

Unsere Arbeit zeigt, wie Methoden der Fraktal- und Topologieanalyse helfen können, die mehrskalige Struktur des interstellaren Mediums zu erfassen und mit physikalischen Prozessen der Sternentstehung zu verknüpfen.

\clearpage