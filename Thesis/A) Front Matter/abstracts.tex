		%Please consider the following section of the "Formvorschriften für die gedruckte Version"
	%Im Anhang ist eine Zusammenfassung (Abstract) mitzubinden. 
	%Ist die Arbeit in einer Fremdsprache verfasst, ist im Anhang jedenfalls eine deutsche Zusammenfassung mitzubinden.
\chapter{Abstract}

Understanding the structure of molecular clouds is crucial for unveiling the physical processes that govern star formation. In this thesis, we present a detailed morphological and topological analysis of the Orion~A and Orion~B molecular clouds, employing tools rooted in fractal geometry and Minkowski functionals. Using column density maps derived from far-infrared dust emission, we apply both global and local perimeter–area techniques to quantify the fractal dimension across a wide dynamic range. 

Our results reveal a clear structural transition in Orion~A at a column density of $N \sim 1.2 \times 10^{22} \,\mathrm{cm}^{-2}$, coinciding with the onset of dense core formation and enhanced star formation activity. Orion~B, in contrast, maintains a scale-free, more turbulent morphology with no such transition. The Euler characteristic further supports this picture, showing a well-defined topological peak in Orion~A, absent in Orion~B. We also introduce a local fractal dimension map and mass–size–dimension (MSD) diagrams, enabling a spatially resolved assessment of structural complexity.

By correlating these structural diagnostics with the distribution of young stellar objects (YSOs), we find that regions of higher fractal dimension tend to host more early-stage protostars, particularly in Orion~A. This suggests a connection between morphological complexity and ongoing star formation. Our findings demonstrate the value of fractal and topological frameworks in disentangling the multiscale physics of the interstellar medium and linking cloud structure to star formation processes.


