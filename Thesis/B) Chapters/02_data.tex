\chapter{Data}
\label{chap:data}

\section{\textit{Herschel} column-density maps}
The main data source for this work comes from ESA's \textit{Herschel} Space Observatory \cite{pilbratt2010herschel}, which was launched with the goal of observing in the far-infrared and submillimetre wavelengths, enabling unprecedented studies of the cold and dusty regions of space where stars and galaxies form.
\textit{Herschel}'s mission started in 2009 and ended in 2013, with huge amounts of data being collected from a variety of instruments.

\textit{Herschel}'s observations of Orion, the target of this work, were part of the \textit{Herschel} Gould Belt Survey (HGBS) , which aimed to map the nearby star-forming regions in the Gould Belt, a flat structure of a few hundred parsecs inclined by $\approx$ 20 $\deg$ to the Galactic plane \cite{andre2010herschel}.

The data used in this work are column-density maps of the Orion A and B molecular clouds, derived from \textit{Herschel} observations. These maps were created by combining data from the Photodetector Array Camera and Spectrometer (PACS) and the Spectral and Photometric Imaging Receiver (SPIRE) instruments, which observed at different wavelengths to capture the dust emission across the clouds.

% Technicalities of the instruments (what do they measure and how, advantages and disadvantages, etc.)

PACS provided imaging photometry in the 60-210 $\mu m$ wavelength range, using two filled silicon bolometer arrays with 16 $\times$ 32 and 32 $\times$ 64 pixels. In photometric mode, PACS simultaneously imaged two bands: either 60-85 $\mu m$ or 85-125 $\mu m$, along with 125-210 $\mu m$, covering a field of view of approximately 1.75 $\times$ 3.5 arcminutes with near-Nyquist sampling \cite{poglitsch2010photodetector}. These bands are particularly sensitive to warmer dust components.

SPIRE extended the coverage to longer wavelengths, better tracing colder dust. Its photometer operated in three broad bands centered at 250 $\mu m$, 350 $\mu m$, and 500 $\mu m$, well-suited for mapping large-scale dust emission from the cold interstellar medium \cite{griffin2010herschel}. 

The combination of PACS and SPIRE photometric data allows for the characterization of the dust spectral energy distribution (SED) across a broad range of temperatures. This broad wavelength coverage enables accurate fitting of modified blackbody models to derive dust temperatures, column densities, and emissivity properties.

% Figure throughputs (\cite{lombardi2014herschel})
% Mabye more on the above once you describe better the process below
\subsection{Derivation of column density maps}
How do you go from what PACS and SPIRE measure to dust emission maps?
Going from dust maps to column density maps... 

Technicalities of the data (good and bad stuff) -> Lombardi et al.
Specific to these maps (coverage and such)

\subsection{Data processing}

The data were pre-processed using the Herschel interactive processing environment (HIPE, Ott 2010) version 10.0.2843, and the latest version of the calibration files.

\section{YSOs}
Extra: YSOs data? TBD

