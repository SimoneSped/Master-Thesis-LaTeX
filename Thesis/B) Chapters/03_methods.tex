\chapter{Methods}
\label{ch:methods}

\section{Introduction}

Fractal Dimension

Minkowski functionals are a set of geometric measures that can be used to quantify the shape and structure of objects in a given space. They are particularly useful in the analysis of complex structures, such as those found in astrophysical data, where they can provide insights into the underlying physical processes.

Oliver's diagrams

Fractal Dimension is a measure of the complexity of a fractal object, which can be defined as the ratio of the change in detail to the change in scale. It is a key concept in fractal geometry and is used to characterize the self-similarity and scaling properties of fractals.

Many ways, focus on Perimeter Area (why?)

Divided into global properties (one number to describe it all) and local properties (how it changes as a function of column density)

\section{Global Properties}

\subsection{Global Fractal Dimension}

perimeter-area (that one paper)

\section{Local Properties}

\subsection{Local Fractal Dimension}

Me 

Dandrogram-like segmentation

\subsection{Mass-Size-Fractal Dimension (MSD) plane}

% maybe also as a 3d plane?

\subsection{Genus}

Genus can help, as it provides a measure of the number of holes in a structure, which can be indicative of the complexity and connectivity of the object. In astrophysical contexts, genus can be used to analyze the topology of structures such as galaxies, clusters, and cosmic filaments.

\section{Connection to star formation}

TBD

\section{Simulations}

Simulation for Global Properties

Simulation for Local Properties 

GRF, simple shapes, etc.