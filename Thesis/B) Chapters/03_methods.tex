\chapter{Methods}
\label{ch:methods}

% to-do:
% define global properties and local properties better and clearer!
\section{Introduction}

The ISM is a complex and dynamic environment, shaped by a broad spectrum of physical processes involving turbulence, gravity, magnetic fields, and stellar feedback. Observationally, the ISM presents a rich variety of morphologies, where the interplay between turbulence and self-gravity leads to intricate patterns. Traditional statistical tools can fall short in fully capturing the spatial complexity of these structures. To address this, the use of topological and morphological descriptors has become increasingly relevant.

In this chapter, we introduce the framework of topological methods used to characterize the structure of the ISM, focusing on the Minkowski functionals and the concept of the fractal dimension. These tools provide a quantitative way to analyze the geometry and topology of cloud structures, offering insight into properties such as size, shape, fragmentation, and connectivity.

In particular, the fractal dimension—derived here through a perimeter–area relation—serves as a proxy for the complexity of a structure’s boundary. It allows us to assess how self-similar or fragmented a region is, providing a direct link between morphology and the physical processes that govern cloud evolution and star formation.

\section{Minkowski Functionals}

The Minkowski functionals are a set of topological measures that can be used to quantify the shape and structures of objects in a space \cite{mecke1993robust}. From these functionals, we can derive various properties of the structures, such as their size, shape, and connectivity.
These can be defined for any dimension, but we will focus on the 2D case, which is relevant for the analysis of the column density maps.

First it is convenient to define the concept of threshold, which is important both in understanding the methods but also in the application.
We define the \textit{excursion set} $S_{\nu}$ of a given field on a $d$-dimensional domain $D \subset R^d$ as the set of points where $f$ exceeds a given threshold parameter $\nu$:

\begin{equation}
    S_{\nu} = \{ \mathbf{x} \in D : f(\mathbf{x}) > \nu \}
\end{equation}

The boundary of the excursion set is defined as the \textit{level set}, which is the set of points where the field equals the threshold:

\begin{equation}
    \partial S_{\nu} = \{ \mathbf{x} \in D : f(\mathbf{x}) = \nu \}
\end{equation}

An example of these can be seen in Figure \ref{fig:level_set_example}, adapted from \cite{hahnLectures}.

\begin{figure}[t]
    \centering
    \includegraphics[width=0.65\textwidth]{figures/boundaries.png}
    \caption{Example of an excursion set and its boundary for a field for a smoothed three-dimensional Gaussian field.}
    \label{fig:level_set_example}
\end{figure}

The motion-invariants are defined as the integrals of the Minkowski functionals over the excursion set $S_{\nu}$, are d+1 and are called the Minkowski functionals. For a 2-dimensional map, these are:

\begin{itemize}
    \item $M_0(S_{\nu})$: the area of the excursion set, which is the integral of the field over the set $S_{\nu}$.
    \item $M_1(S_{\nu})$: the perimeter of the excursion set.
    \item $M_2(S_{\nu})$: the Euler characteristic of the excursion set, which is a measure of the connectivity of the set.
\end{itemize}

From these, we can define a lot of properties that can be calculated to characterize the structures in Orion A and B. 

The following section is divided into two parts: global and local properties.  
\textbf{Global properties} refer to statistical measures that characterize the cloud as a whole—capturing the average morphological behavior across all column density scales.  
In contrast, \textbf{local properties} probe how structural complexity evolves as a function of column density, offering insight into spatial variations and scale-dependent behavior within the cloud.

\section{Global Properties}

\subsection{Global Fractal Dimension}

The fractal dimension is a measure of the border complexity of a given structure. This can be calculated using the Minkowski functionals, specifically the perimeter and area \cite{cannon1984fractal}. 
The main idea for this approach is that, for a given border length, a simple border will always enclose a larger area than a complex border.

On this idea, the perimeter-area (P-A) relation is defined as:

\begin{equation}
    \label{eq:perimeter_area}
    P \propto A^{\frac{D}{2}}
\end{equation}

By measuring the perimeter $P$ and area $A$ of a structure at different thresholds and fitting the logarithm of the perimeter to the logarithm of the area, we can derive the fractal dimension $D$ of the overall structure. Hence this characterized in this work as a \textbf{global property} of the cloud, and it's called global fractal dimension.

The scaling relationship between perimeter and area offers insight into the structural complexity of a shape’s boundary. Intuitively, a smoother and more compact boundary can enclose more area with less perimeter compared to a more irregular or fragmented contour. This variation in complexity is quantified through the global fractal dimension, \( D \), which describes how the perimeter scales with area.

For simple geometries—like circles or squares—the perimeter \( P \) scales with the square root of the area \( A \), following \( P \propto A^{1/2} \), yielding \( D = 1 \), as expected for a one-dimensional boundary. In contrast, for highly intricate and space-filling contours, the perimeter scales more directly with the area, i.e., \( P \propto A \), corresponding to a fractal dimension approaching \( D = 2 \), the value expected for a boundary that becomes increasingly space-filling \cite{lovejoy1982area}.

When a structure includes components with distinct spatial scales—for example, large coherent features overlaid with fine-grained substructure—these different regimes may exhibit different perimeter–area scaling behaviors. In such cases, the presence of multiple scaling slopes (or values of \( D \)) in different regimes can reveal the existence of characteristic length scales, hinting at underlying physical processes operating differently at those scales.

On the other hand, if the data show a consistent scaling behavior across the full range of column densities—with the perimeter and area following a single power-law in log–log space—this suggests scale-free structure and supports the interpretation that the cloud morphology is fractal in nature. An example of this is illustrated in Figure~\ref{fig:perimeter_area_example}, where data from meteorological clouds were used to derive the fractal dimension of the structures. The slope of the linear fit yields the fractal dimension $D$, providing a quantitative measure of the structural complexity. The lack of a characteristic scale, evidenced by the good fit, suggests that turbulent processes are likely responsible for shaping the observed structures.

\begin{figure}[t]
    \centering
    \includegraphics[width=0.4\textwidth]{figures/lovejoy.png}
    \caption{Example of the perimeter-area relation for meteorological clouds \cite{lovejoy1982area}. The slope of the linear fit gives the fractal dimension of the structure.}
    \label{fig:perimeter_area_example}
\end{figure}

The perimeter-area relation is a powerful tool for analyzing the complexity of structures in the ISM. It's simple, yet effective, and can be applied to a wide range of structures, from clouds to galaxies. The fractal dimension derived from the perimeter-area relation can provide insights into the physical processes that shape these structures, such as turbulence, gravity, and magnetic fields.
However, it is important to note that the perimeter-area relation is not always applicable to all structures. Furthermore, numerical effects can also play a role in the calculation of the fractal dimension, as we will see in the simulations section. Artifacts also have an effect on this method \cite{imre2006artificial}. This is to say, that the method has its limitations and simulations play an important role in understanding these effects. 

\section{Local Properties}

\subsection{Local Fractal Dimension}

One further advantage of the perimeter–area relation is that it can be inverted to derive the fractal dimension of a structure at a given column density threshold. By inverting Eq.~\ref{eq:perimeter_area}, we can express the local fractal dimension $D$ as a function of the column density threshold $\nu$:

\begin{equation}
    \label{eq:local_fractal_dimension}
    D(\nu) = 2 \times \frac{\log\bigl(P(\nu)\bigr)}{\log\bigl(A(\nu)\bigr)}.
\end{equation}

This approach enables a more detailed characterization of the geometry and scaling behavior of the cloud structures as a function of physical conditions. In particular, by applying a series of increasing thresholds $\nu$, one effectively isolates progressively denser substructures within the molecular cloud. At each threshold, the perimeter $P(\nu)$ and area $A(\nu)$ can be measured either for all connected structures collectively or for each identified region individually.

In the first case, the perimeters and areas of all structures present in the map at a given threshold are summed, providing a global measurement of the local fractal dimension for that threshold. This yields a single $D(\nu)$ value for each column density level, capturing how the overall morphological complexity evolves as successively denser material is selected. An example of this approach is shown in Figure~\ref{fig:local_fractal_dimension_example}, where the perimeters and areas are computed across a range of thresholds, demonstrating the dependence of the fractal dimension on the chosen column density cut.

\begin{figure}[t]
    \centering
    \includegraphics[width=0.45\textwidth]{figures/example_calculations.png}
    \caption{Simplified example of the calculation of perimeters and areas of structures at a given column density threshold. The local fractal dimension can be calculated by summing over all of the structures at that threshold, or for each of the single structures.}
    \label{fig:local_fractal_dimension_example}
\end{figure}

Conversely, it is also possible to calculate the local fractal dimension separately for each individual structure as a function of threshold. In this case, the evolution of $D(\nu)$ with increasing density can be tracked on a per-object basis, providing insights into whether substructures become more or less fragmented and irregular as higher-density material is isolated. This allows the study of whether certain regions of the cloud exhibit self-similar scaling over a wide range of column densities, or whether there are transitions in the structural properties that may reflect different physical regimes, such as turbulence, gravitational collapse, or feedback from star formation.
Because this method analyzes structures as a function of column density, it is considered a \textbf{local property}. Moreover, as this approach is relatively novel, its interpretation must be carefully developed based on the results obtained in this work.

Analyzing the local fractal dimension as a function of threshold thus provides a tool to investigate the scale dependence and complexity of molecular clouds, complementing the global measurements and helping to disentangle the contributions from different physical processes.

% to-do: a picture here would not be bad
\subsection{Dendrograms}

In order to explore the segmentation of structures with changing column density or mass, dendrograms are a useful tool. A dendrogram is a diagram that represents the hierarchical organization of structures within the data \cite{everitt2010cambridge}. Specifically, it shows how large-scale regions can be decomposed into smaller substructures as the threshold for the column density (or mass) increases.

By progressively applying higher thresholds, one can trace how individual dense clumps emerge from the diffuse background and how these clumps further fragment into even denser cores. In this way, dendrograms provide a visual and quantitative framework to analyze the nested, multiscale nature of molecular cloud structure. An example of a resulting tree for an example shape can be seen in Figure \ref{fig:dendro} (from the Astrodendro Package \footnote{\href{https://dendrograms.readthedocs.io/en/stable/}{Astrodendro}}).

\begin{figure}[t]
    \centering
    \includegraphics[width=0.45\textwidth]{figures/dendrogram.png}
    \caption{Simplified example of a resulting tree/branch/leave structure from the Astrodendro package.}
    \label{fig:dendro}
\end{figure}

In this work, dendrograms were constructed using hierarchical clustering methods applied to the mass maps of Orion A and B. This approach allows identification and characterization of the main branches (large structures) as well as the terminal leaves (smallest resolved substructures). For each identified region, properties such as total mass, size (computed via principal component analysis), and fractal dimension were derived to quantify their physical characteristics and complexity.

\subsection{Mass-Size-Fractal Dimension (MSD) plane}

Since it is possible to calculate the local fractal dimension for single structure, we can connect this property to more physical aspects of the substructures, namely mass and size. This then allows us to include new information on already established mass-size diagrams \cite{Hacar_2025}, creating a Mass-Size-Fractal Dimension (MSD) plane. The mass and size of each structure are computed using its binary mask: the mass is obtained by summing the corresponding pixels in the mass map, while the size is estimated by performing a principal component analysis (PCA) on the mask to determine its spatial extent.

The dendrogram-like segmentation is an important aspect of this process, as it ensures a proper characterization and recognition of the sub-structures. 

\subsection{Euler characteristic}

Euler characteristic, often denoted as $\chi$ (or $M_2(S_{\nu})$ in the context of the Minkowski functionals), and in two dimensions it is given by:
\begin{equation}
    \chi = \text{Number of connected regions} - \text{Number of holes}
\end{equation}
For a given excursion set $S_{\nu}$, the Euler characteristic provides a measure of the topology of the structures, indicating how many isolated regions and holes are present at a specific threshold. In practice, a positive Euler characteristic indicates more isolated regions than holes, while a negative value suggests a topology dominated by holes.

The Euler characteristic is closely related to the genus $G$ of the structure, where in two dimensions $G = 1 - \chi$. The genus is commonly used in cosmology and astrophysics to describe the connectivity of structures, such as the filamentary network in the ISM or the cosmic web. By analyzing the Euler characteristic or genus as a function of the threshold, one can gain insights into the morphological transitions and connectivity of the observed structures.

Since this is also a property that can be expressed as a function of the column density thresholds $\nu$, it also counts towards the \textbf{local properties} of the cloud.
\section{Connection to Star Formation}

To investigate the link between cloud structure and star formation activity, we employed the catalogue of young stellar objects (YSOs) compiled by \cite{megeath2012catalogue}.  
This catalogue provides both positional information and evolutionary classifications of YSOs within Orion~A and Orion~B, allowing a direct spatial comparison with the structures identified in our analysis.

The degree of association between the local structural properties and the YSO distribution was quantified using the Pearson correlation coefficient \cite{pearson1895vii} and the Spearman rank correlation coefficient \cite{sedgwick2014spearman}.  
These metrics capture linear and monotonic relationships, respectively, and together provide a robust measure of how variations in the structural parameters relate to the spatial density of YSOs.

We restricted the YSO sample to objects classified as Class I or flat-spectrum sources, based on their infrared slope (\texttt{alpha}) and the catalogue classification flag (\texttt{Cl}). These stages are closely associated with active star-forming regions, while more evolved Class II/III sources may have migrated from their birth sites and were therefore excluded \cite{lada1987star}. For comparison, we also computed correlations using the full YSO sample.

\section{Simulations}

\subsection{Simulations on the Global Properties}

The simulations addressing the global properties are primarily designed to verify the interpretation of the values obtained for the global fractal dimension $D$, as described in the preceding sections. In particular, they aim to confirm the expectation that the perimeter–area relation should yield unreliable or systematically biased results for structures that possess a well-defined characteristic length scale. 

In this context, Gaussian Random Fields (GRFs) play an important role as a benchmark for comparison. GRFs allow the generation of synthetic structures with controlled statistical properties and scaling behavior, providing a framework to test whether the methodology is sensitive to the presence or absence of intrinsic scales.

\subsection{Simulations on the Local Properties}

Analogously, simulations are employed to better understand how different values of the local fractal dimension arise in different types of structures. These tests include the analysis of Gaussian Random Fields, purely Gaussian noise, and simple geometric shapes such as lines and circles. 

The simulations also allow for quantifying numerical effects and resolution-dependent biases, assessing how limited spatial resolution and discretization influence the measured properties of the identified structures. This helps ensure that the interpretation of the fractal dimension is robust and not merely an artifact of the analysis method or data sampling.

\subsection{Comparison with Other Methods}

In the context of the simulations, a brief comparison with alternative approaches was carried out. This served both to improve understanding of the fractal dimension framework and to identify potential limitations or artifacts of the perimeter–area (PA) method.

The method most frequently used for comparison was the box-counting technique, which relies on covering the shape with a grid of boxes of varying sizes and counting the number of boxes that contain part of the shape. By examining how this count scales with the box size, the fractal dimension can be estimated \cite{falconer2013fractal}. 

\section{Uncertainties}

% Probs need to describe how to go from sigma_P and sigma_A to sigma_D ...
\subsection{Perimeter and Area}

The measurement of perimeter and area is subject to inherent uncertainties. To estimate these uncertainties, the following procedure was carried out:

\begin{itemize}
    \item The perimeter ($P$) and area ($A$) are measured for a set of circles with known true perimeter and area.
    \item The magnitude of the measurement error are computed by comparing the measured values to the true values.
    \item The errors are averaged over a sufficiently large number of samples ($N$) to obtain representative estimates of the typical deviation.
\end{itemize}

\subsection{Beam Size}

The finite resolution imposed by the telescope beam introduces systematic uncertainties in the measurement of structural properties. The beam convolution smooths small-scale fluctuations and reduces the complexity of the contour at each column density threshold. This effect generally leads to an underestimation of the perimeter and an overestimation of the area, biasing the derived fractal dimension toward lower values.

To quantify this uncertainty, one can simulate synthetic column density maps with known fractal characteristics and convolve them with a Gaussian kernel matching the observational beam. By comparing the perimeter–area relations before and after convolution, the bias introduced by the finite resolution can be estimated. Alternatively, the characteristic beam size provides a natural lower limit to the spatial scales where fractal analysis is reliable. In this work, scales smaller than 2–3 times the beam full-width at half-maximum (FWHM) are treated with caution, and the derived fractal dimensions are interpreted as lower limits to the intrinsic complexity of the cloud morphology.

\subsection{Uncertainties in the Extinction Map}

The dataset also includes a pixel-wise $\tau_{850}$ error map, as shown in Figure \ref{fig:error_map}. 

\begin{figure}[t]
    \centering
    \includegraphics[width=0.75\textwidth]{figures/error_map.png}
    \caption{Pixel-wise error map of the $\tau_{850}$ data \cite{lombardi2014herschel}}
    \label{fig:error_map}
\end{figure}

Although these uncertainties are of a non-negligble magnitude to other sources of error and, in principle, all effects should be taken into account, the dominant contributions to the total uncertainty arise from the measurements of perimeter and area. 
For this reason, the pixel-wise extinction errors are neglected in the following analysis.
