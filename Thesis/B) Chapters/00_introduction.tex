\chapter{Introduction}

Giant Molecular Clouds (GMCs) are the primary sites of star formation in galaxies. Within these massive, cold, and dense regions, stars form from the gravitational collapse of the highest column density substructures, such as filaments and cores. Understanding the morphology and internal structure of GMCs is therefore essential to understanding the conditions that regulate star formation.

Much of our knowledge of GMC structure comes from the study of dust emission. Dust grains—well-coupled to the gas in these clouds—emit thermal radiation in the far-infrared and submillimeter regimes. These emissions can be used to reconstruct column density maps, offering a valuable window into the distribution of material within the cloud.

To go beyond qualitative visual interpretation, quantitative tools are needed to describe the spatial complexity of these structures. One powerful approach involves applying methods from fractal geometry and topology. The concept of a fractal dimension, first introduced by Mandelbrot, provides a way to quantify the irregularity or complexity of a structure's boundary. Complementary topological measures, such as the Euler characteristic and Minkowski functionals, allow us to describe properties like connectivity and fragmentation.

Historically, various techniques have been employed to estimate the fractal dimension of molecular clouds—from simple box-counting algorithms to power-law scalings of structure size distributions. However, these methods often provide a limited or averaged view of complex, hierarchical structures.

The motivation behind this thesis is to develop and apply tools that connect the visual complexity of molecular clouds with physically meaningful, measurable quantities. By refining the analysis of fractal and topological properties, we aim to gain deeper insight into the role of turbulence, structure formation, and star formation in GMCs.

The goals of this thesis are:
\begin{itemize}
    \item to extract and preprocess observational column density data for selected GMCs,
    \item to characterize their structure using topological methods such as the Euler characteristic and perimeter–area scaling,
    \item to explore the fractal dimension framework in both global and local forms, and
    \item to investigate correlations between structural complexity and star formation indicators.
\end{itemize}

The thesis is organized as follows:
\begin{itemize}
    \item \textbf{Chapter 2} provides theoretical background on molecular clouds, fractal geometry, and the topological tools used throughout this work.
    \item \textbf{Chapter 3} describes the datasets employed, including how column density maps and YSO catalogs are used.
    \item \textbf{Chapter 4} outlines the methods used to calculate fractal and topological descriptors, with a focus on perimeter–area relations and dendrogram-based analysis.
    \item \textbf{Chapter 5} presents the results, including fractal dimension maps, topological behavior, and the relationship between cloud structure and star formation.
    \item \textbf{Chapter 6} discusses the implications of the results and compares them with previous literature and theoretical expectations.
    \item \textbf{Chapter 7} summarizes the main findings and outlines possible directions for future research.
\end{itemize}

Additional material, including simulation details and visual galleries of structures, is provided in the Appendices.
